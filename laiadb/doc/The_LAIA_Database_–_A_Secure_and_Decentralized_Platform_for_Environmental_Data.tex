\documentclass[conference,final,]{IEEEtran}


\usepackage{graphicx}
% We will generate all images so they have a width \maxwidth. This means
% that they will get their normal width if they fit onto the page, but
% are scaled down if they would overflow the margins.
\makeatletter
\def\maxwidth{\ifdim\Gin@nat@width>\linewidth\linewidth
\else\Gin@nat@width\fi}
\makeatother
\let\Oldincludegraphics\includegraphics
\renewcommand{\includegraphics}[1]{\Oldincludegraphics[width=\maxwidth]{#1}}

\usepackage[unicode=true]{hyperref}

\hypersetup{
            pdftitle={The LAIA Database -- A Secure and Decentralized Platform for Environmental Data},
            pdfborder={0 0 0},
            breaklinks=true}
\urlstyle{same}  % don't use monospace font for urls

% Pandoc toggle for numbering sections (defaults to be off)
\setcounter{secnumdepth}{0}

% Pandoc syntax highlighting

% Pandoc header

\providecommand{\tightlist}{%
  \setlength{\itemsep}{0pt}\setlength{\parskip}{0pt}}

%% END MY ADDITIONS %%


\hyphenation{op-tical net-works semi-conduc-tor}

\begin{document}
%
% paper title
% Titles are generally capitalized except for words such as a, an, and, as,
% at, but, by, for, in, nor, of, on, or, the, to and up, which are usually
% not capitalized unless they are the first or last word of the title.
% Linebreaks \\ can be used within to get better formatting as desired.
% Do not put math or special symbols in the title.
\title{The LAIA Database -- A Secure and Decentralized Platform for
Environmental Data}

% author names and affiliations
% use a multiple column layout for up to three different
% affiliations

\author{

%% ---- classic IEEETrans wide authors' list ----------------
 % -- end affiliation.wide
%% ----------------------------------------------------------



%% ---- classic IEEETrans one column per institution --------
 %% -- beg if/affiliation.institution-columnar
\IEEEauthorblockN{Saulo Jacques
 %% -- end for/affiliation.institution.author
}
\IEEEauthorblockA{Cal Teixidor
\\saulojacques@protonmail.com
}
 %% -- end for/affiliation.institution
 %% -- end if/affiliation.institution-columnar
%% ----------------------------------------------------------





%% ---- one column per author, classic/default IEEETrans ----
 %% -- end if/affiliation.institution-columnar
%% ----------------------------------------------------------

}

% conference papers do not typically use \thanks and this command
% is locked out in conference mode. If really needed, such as for
% the acknowledgment of grants, issue a \IEEEoverridecommandlockouts
% after \documentclass

% for over three affiliations, or if they all won't fit within the width
% of the page, use this alternative format:
%
%\author{\IEEEauthorblockN{Michael Shell\IEEEauthorrefmark{1},
%Homer Simpson\IEEEauthorrefmark{2},
%James Kirk\IEEEauthorrefmark{3},
%Montgomery Scott\IEEEauthorrefmark{3} and
%Eldon Tyrell\IEEEauthorrefmark{4}}
%\IEEEauthorblockA{\IEEEauthorrefmark{1}School of Electrical and Computer Engineering\\
%Georgia Institute of Technology,
%Atlanta, Georgia 30332--0250\\ Email: see http://www.michaelshell.org/contact.html}
%\IEEEauthorblockA{\IEEEauthorrefmark{2}Twentieth Century Fox, Springfield, USA\\
%Email: homer@thesimpsons.com}
%\IEEEauthorblockA{\IEEEauthorrefmark{3}Starfleet Academy, San Francisco, California 96678-2391\\
%Telephone: (800) 555--1212, Fax: (888) 555--1212}
%\IEEEauthorblockA{\IEEEauthorrefmark{4}Tyrell Inc., 123 Replicant Street, Los Angeles, California 90210--4321}}




% use for special paper notices
%\IEEEspecialpapernotice{(Invited Paper)}




% make the title area
\maketitle

% As a general rule, do not put math, special symbols or citations
% in the abstract
\begin{abstract}
The current centralized storage of data in private servers owned by big
corporations exposes the whole society to the threats of surveillance,
censorship and malicious attacks coming from different sources. Under
this context, the environmental and climate data are also a target of
these threats, therefore a shift on the topology of data storage and
sharing is needed. The present article proposes the LAIA Database, a
decentralized and encrypted platform focused on environmental and
climate data that provides a total control over the data by the users.
LAIA is a friend-to-friend network (F2F), namely it only allows access
to the data by trusted nodes on the network in a secure fashion,
protected from malicious attacks. The project aims to improve the
research collaboration between groups as it allows to share important
information between trusted parts, while it ensure the security and freedom
of both research groups in a trustworthy network.
\end{abstract}

% no keywords

% use for special paper notices



% make the title area
\maketitle

% no keywords

% For peer review papers, you can put extra information on the cover
% page as needed:
% \ifCLASSOPTIONpeerreview
% \begin{center} \bfseries EDICS Category: 3-BBND \end{center}
% \fi
%
% For peerreview papers, this IEEEtran command inserts a page break and
% creates the second title. It will be ignored for other modes.
\IEEEpeerreviewmaketitle


\hypertarget{introduction}{%
\section{Introduction}\label{introduction}}

The concepts of privacy, anonymity and digital self-defense are not only
restricted to activists and groups directly involved with digital rights
and other political movements. Since the surveillance apparatuses of the
state were revealed in detail by Edward Snowden, many social movements
started to deal with technological devices in a different way. Back
then, the massive storage of data in a centralized database under US
jurisdiction came out to the public spot.The issue of mass surveillance
not only affected citizens but also other nation-states depending on
USA-based infrastructures, which eventually turned out into an
international crisis (Sargsyan 2016).

Recently there has been a rise of critical voices against investments in
environmental and climate research programs. For some of them,
environmental protection is seen as a threat to industries and the
development of the country, this is something on which many politician
base their rhetoric (Trump, for instance, as a climate-change-denier
based his campaign on this issue with the slogan ``make America great
again''). Once in power conservative politicians supported by
climate-change-deniers groups has moved from a mere rhetoric into
political actions across the globe (De Pryck and Gemenne 2017). This new
political scenario resulted in a strong reaction from researchers and
activists that started to take some measures in order to protect the
data from being altered, deleted or removed ( an example of this
measures was the mass-download of US government climate data before
Donald Trump's administration took over). Nonetheless, what it is seen
is a shift of data storage from US-based servers to servers under
different jurisdiction but owned by private groups and big corporations
keeping high centralization of the knowledge on the Web in a few large
providers (Halpin and Monnin 2016). Hence, although this action intends
to rescue the data, its strategy is very ineffective as the political
scenario is constantly changing in accordance to centralized interests.

A more efficient and low cost way to protect the data is based on a real
paradigmatic shift on data storage, from a centralized and vulnerable
structure to a decentralized and serverless architecture. This is a path
to recover the autonomy and control over environmental data by reducing
the risks of data loss, the single point of failure and the exploitation
of centralized data controlled by corporations and governments. This
distributed structure is also social, where the autonomy is associated
to the strength social bonds and flux of information. Even though there
is already some decentralized projects available (e.g BitTorrent, git,
etc), none of them was developed focusing on climate and environmental
data.

The current improvement of environmental data generation, from
universities and research institutes to citizen scientific labs and
collectives, contrasts with the lack of an integrated platform
infrastructure where the data is published and the communication between
different groups involved with the environmental monitoring is enhanced.
Concerning to the aforementioned limitations and vulnerabilities of the
mainstream database structures, this project aims to develop an
accessible and decentralized database based on trustworthy networks,
privacy and anonymity. Under this decentralized structure a central
service is not required and if a node is under attack, the other parts
of the network can work without interruption.

Serverless means that each node is server\&client at the same time,
since no external service is needed for the network to function. Thus,
it's based on the principle of a Network of Participation, which means
that the data will never disappear as long as there is a network of
people interested on keep it available in their computer. Moreover, with
the multi-source sharing (swarming) and the multi-hop a small number of
hops among nodes is expected to access files resulting on faster
downloads (low latency).

\hypertarget{the-database-architecture}{%
\section{The Database Architecture}\label{the-database-architecture}}

The LAIA Database is a cross-platform database focused on the user
experience in order to build up a Web of Trust between peers. It
integrates a strong encrypted protocol with usability and gives a total
control over the data by users, which is shared only with trusted nodes.
In the decentralized structure of LAIA database the data is hosted in
swarms, where a user who join the swarm become a peer and can download
pieces of the data from other node since establish connection to at
least one trusted peers in the swarm. In return, the user uploads pieces
of data they already have to trusted peers who need it. Once the user
have all of the data, they can choose to remain in the swarm and
continue sharing with other trusted peers, which makes them a seed.
Accordingly, in this Network of Participation the structure, the more
popular the data, the bigger the swarm, and the faster the downloads.

LAIA database is a friend-to-friend (F2F) network, a private
peer-to-peer in which each network node exchanges data only with a
designated list of ``friend'' nodes. A F2F network builds its overlay on
top of pre-existent trust relationships among its users. A digital life
is a dimension of real-life and this is a very important base of LAIA's
principles where only trusted nodes establish connection among them,
resulting in a significantly more secure network. With F2F it is
possible to ensure a cryptographically secure connection established
between nodes that had met each other in a past experience outside the
platform. Thus, unlike peer-to-peer (P2P) network, with F2F it is
possible to control who accesses each others data through encrypted
connections between trusted parts that had already shared secrets needed
to establish these secure connections.

Related to the search for files, it is possible for nodes to look for
files accessing a large network even when still only connected to their
trusted nodes. The trusted-node search works by propagating search and
file transfers recursively to all friends of friends to a certain degree
of separation. This topology ends up in a non-homogeneous structure with
regard to the numbers of nodes and bandwidth (Fig. 1), where the
swarming structure ensure either technical or legal means to overcome
traditional ways for limiting the access to the Web (Fig. 2).

The LAIA database uses the reputation scores to give a general overview
of the peers' level of participation in the network. The role of this
tool is to protect nodes against the problems of P2P network with spams
and malicious nodes, preventing disturbing data spreading without
control. Reputation score is accounted for each node considering
negative score as bad, positive as good and zero as neutral. If the
score is too low, the identity is flagged as bad. During the
bootstrapping process for joining the P2P network, peers can potentially
use reputations to decide who to directly connect to in the overlay
topology.

\begin{figure}
\centering
\includegraphics{network.png}
\caption{Friend-to-Friend: The network of no-homogeneous networks. As
the swarming is built up based on trust between nodes, the networks
might vary in matter of bandwidth and size}
\end{figure}

\begin{figure}
\centering
\includegraphics{networkprivacy.png}
\caption{Privacy and distribution of data throughout the network: Only
trusted nodes (blue dots) can access the data from the source (green
dots), while nodes ``friend-of-friend'' (red dots) can't see the data}
\end{figure}

\hypertarget{security-and-privacy}{%
\subsection{Security and Privacy}\label{security-and-privacy}}

The creation of a profile is actually the generation of a PGP key, where
a password is used to to validate the profile. At the login, the PGP key
is used to decrypt the encrypted SSL certificate and its password is
read. The SSL password is chosen randomly when creating the location and
encrypted using the users' PGP key.

The data is identified by their specific fingerprint, and unlike other
decentralized projects (e.g.~IPFS) it is also possible to look for files
by their cryptographic hashes, names or even by their sizes. The
security relies on cryptographic algorithms in which the connection
between nodes are encrypted using SSL, while the identity of node is
represented by its PGP key. Once the software is installed and started
for the first time, a profile and a certificate is automatically
generated. That certificate, in turn, is used to authenticate and
connect safely with the trusted node to include in the network.

Once a folder is added to be shared, all the files present inside will
be hashed (i.e.~there will be created a fingerprint for each file)
before being shared in the program. This folder can be shared as
``browsable'', in which only trusted nodes can see and download the
shared files, ``network wide'', where friends of friend can download the
files via anonymous tunnels. In this case, the friends of friends nodes
are not capable to see which files or node is being shared, but they can
find the files using the search.

\hypertarget{a-using-multiples-accounts-for-a-group}{%
\subsubsection{a) Using Multiples Accounts For a
Group}\label{a-using-multiples-accounts-for-a-group}}

LAIA allows to use an existing PGP key for signing multiple identities,
and thus makes team work easy. As detailed before, the PGP keys will be
used for encrypting the SSL passphrase on disk and to sign the
location's SSL certificate.

\hypertarget{b-files-hashing}{%
\subsubsection{b) Files Hashing}\label{b-files-hashing}}

A way LAIA stores files is versioning them through the 40-character
secure hash algorithm SHA-1. In 2017, some concerns came out since
researchers had announced a hash collision of the SHA-1. A hash
collision occurs when two separate inputs produce the same hash result
output.

Although a has collision is unlikely to occur , there is an important
difference between using cryptographic hash for security signing, and
using it for generating a file content identifier. In the case of a hash
used for security it is used to trust users in order to protect people
using a given platform from untrusted users. On the other hand, in
content-addressable systems, like in LAIA, the hash for content
identifier is not used to trust users but, as previously said, for file
content identification.

\hypertarget{c-estabilishing-a-secure-connection}{%
\subsubsection{c) Estabilishing a secure
connection}\label{c-estabilishing-a-secure-connection}}

LAIA's privacy is protected with anonymous tunnels, the swarms often
consist of computers distributed around the world (in which national
laws cannot actually achieve the censorship), and its structure is
independent from a centralized server controlled by corporations or
nation-states. Thus, there is no single entity to sue or pressure
financially, and the data sharing overcomes the infrastructure
jurisdiction and implication in giving users data to intelligence
agencies.

The secure connection of two nodes is done using the PGP signature that
authenticates SSL links between them. In its structure, only the trusted
nodes can access details like each other's IP addresses, which files are
being shared, etc. One way to improve the privacy and anonymity is using
the Retro-Tor tool that make it possible to run LAIA over the Tor
network, where IP is not visible even to connected nodes. An advantage
for users is that it creates a hidden node with one click away that
enhances the secure data sharing, and overcomes digital embargo or
censorship. This is an important tool for activities that require a
secure and private contact with a trusted persons, being able to send
large files, forum posts, or have a multi-party discussion.


\hypertarget{the_laia-authority-and-the-online-platform}{%
\section{The LAIA Authority and the Online Platform}
\label{the_laia-authority-and-the-online-platform}}

After that a profile is created, a PGP key is automatically exchange
with the LAIA authority. The authority is the only node that has access
to the whole network and the functionality is used to enable the use of
the data at a web interface or app for it access on a integrative platform.

The authority is unable to influence on the reputation system or on the 
ban of nodes, its role is only to access the swarm's data in order to feed online 
platforms. This online tools can be useful to integrate data from different groups 
keeping a strong reliability based on the reputation indexes.



\hypertarget{conclusion}{%
\section{Conclusion}\label{conclusion}}

The LAIA database proposes a new efficient, strong and secure
alternative to the centralized structure of environmental data storage.
The project has many advantage comparing to both centralized and
decentralized platforms. The motivations of the project, based on
autonomy, social participation, and freedom of speech are align to the
essential principles for the improvement of researches and the social
role of science.

The reputation system, as already pointed out by other projects, is an
effective system of self-management, since the majority of the nodes in
the networks are interested to maintain the system stable and free from
trolls and distractions. Furthermore it is faster, safer and ensures
that the data will never disappear as there is at least one node hosting
it. The data is private until a node allows a trusted one to access to
its data. It is also easier to overcome the problem with human friendly
names for files and enhances privacy using Tor network.

The F2F network used by the project is a very strong concept which
allows private sharing of data, and protects data providers, data
consumers as well as intermediaries from the potential negative
consequences stemming from being identified as participants in this data
exchange. Thus, in LAIA Database the control of access to the file is
not used to restrict the access of information, but to protect
researchers groups, collectives connected to the network, as well as the
knowledge itself. It allows data sharing between trusted nodes by using
a free and open source network with high scalability enhancing
the participation of universities, citizen scientists and autonomous
researches in the network.

\newpage

\hypertarget{references}{%
\section*{References}\label{references}}
\addcontentsline{toc}{section}{References}

\hypertarget{refs}{}
\leavevmode\hypertarget{ref-de_pryck_denier_2017}{}%
De Pryck, Kari, and François Gemenne. 2017. ``The Denier-in-Chief:
Climate Change, Science and the Election of Donald J. Trump.'' \emph{Law
and Critique} 28 (2): 119--26.
\url{https://doi.org/10.1007/s10978-017-9207-6}.

\leavevmode\hypertarget{ref-halpin_decentralization_2016}{}%
Halpin, Harry, and Alexandre Monnin. 2016. ``The Decentralization of
Knowledge: How Carnap and Heidegger Influenced the Web.'' \emph{First
Monday} 21 (12). \url{https://doi.org/10.5210/fm.v21i12.7109}.

\leavevmode\hypertarget{ref-sargsyan_data_2016}{}%
Sargsyan, Tatevik. 2016. ``Data Localization and the Role of
Infrastructure for Surveillance, Privacy, and Security.''
\emph{International Journal of Communication} 10 (0): 17.
\url{https://ijoc.org/index.php/ijoc/article/view/3854}.

\end{document}


